\documentclass{article}
\usepackage[utf8]{inputenc}
\usepackage{amsmath}
\usepackage{amssymb}
\usepackage{makeidx}
\title{Fondamenti di Automatica}
\author{Davide Calabrò}

\newcommand{\arcangle}{%
	\mathord{<\mspace{-9mu}\mathrel{)}\mspace{2mu}}%
}

\begin{document}
	\maketitle
	\pagenumbering{gobble}	
	
	\newpage
	\section*{Appunti}
	\subsection{Calcolo esplicito dei movimenti}
	\paragraph{Sistemi L.T.I. a tempo continuo - Formula di Lagrange}
	\begin{equation}
		x(t) = x(t_0)e^{a(t-t_0)} + \int_{t_0}^t e^{a(t-\tau)} b u(\tau) d\tau
	\end{equation}
	\paragraph{Sistemi L.T.I. a tempo discreto}
	\begin{equation}
		y(t) = CA^tX_0 + C \sum_{i=1}^{t} A^{t-i} Bu(i-1) + Du(t)
	\end{equation}
	
	\subsection{Analisi di stabilità del polinomio caratteristico}
	\subsubsection{Sistemi L.T.I. a tempo continuo}
	Dato il polinomio caratteristico
	\begin{equation}
		p(\lambda) = \det(A - \lambda I) = a_0\lambda^n + a_1 \lambda^{n-1} + ... + a_n
	\end{equation}
	allora
	\subparagraph{Condizione necessaria perl'asintotica stabilità}
	Tutti gli $a_i$ hanno lo stesso segno (condizione necessaria e sufficiente se $p(\lambda)$ ha ordine 2)
	\subparagraph*{Criterio di Routh-Hurwitz}
	\begin{enumerate}
		\item Definizione della tabella di Routh
			\begin{equation}
				\begin{bmatrix}
					a_0 && a_2 && a_4 && ... \\
					a_1 && a_3 && a_5 && ... \\
					h_1 && h_2 && h_3 && ... \\
					k_1 && k_2 && k_3 && ... \\
					l_1 && l_2 && l_3 && ...\\
				\end{bmatrix}
			\end{equation}
			Definendo $l_i$ come:
			\begin{equation}
				l_i = -\frac{1}{k_1} \det\left(\begin{bmatrix}
												h_1 && h_{i+1} \\ k_1 && k_{i+1}
											\end{bmatrix}\right)
			\end{equation}
			E' necessario proseguire fino alla riga $n+1$. Inoltre se $\exists k_1 = 0$ non è possibile definire la riga successiva, dunque la tabella non è ben definita.
		\item Il sistema è ASINTOTICAMENTE STABILE sse
			\begin{itemize}
				\item La tabella di Routh è ben definita
				\item Tutti gli elementi della prima colonna hanno lo stesso segno				
			\end{itemize}
			Inoltre il numero di cambiamenti di segno è uguale al numero di autovalori con parte reale positiva.
	\end{enumerate}

	\subsubsection{Sistemi L.T.I a tempo discreto}
	Dato il polinomio caratteristico
	\begin{equation}
		p_A(\lambda) = \det(\lambda I - A) = a_0\lambda^n+...+a_{n-1}\lambda+a_n
	\end{equation}
	\subparagraph{Teo 1} Condizione necessaria è che $\left|\frac{a_n}{a_0}\right| < 1$ e $\left|\frac{a_1}{a_0}\right| < n$
	\subparagraph{Teo 2} Condizione necessaria è che $a_0p_A(\lambda = 1) = a_0(a_0+...+a_n) > 0$
	\subparagraph{Teo 3} Condizione sufficiente è che $a_0>a_1>...>a_{n-1}>a_n$
	\subparagraph{Teo 4} Condizione sufficiente è che $\sum_{i=1}^n \left|a_i\right| < \left|a_0\right|$
	
	\subparagraph{Criterio di Jury}
	E' condizione necessaria e sufficiente
	\begin{enumerate}
		\item Costruzione della tabella
			\begin{equation}
				\begin{bmatrix}
					a_0 && a_1 && ... && a_{n-1} && a_n \\
					h_1 && h_2 && ... && h_\nu \\
					l_1 && l_2 && ... \\
				\end{bmatrix}
			\end{equation}
			con
			\begin{equation}
				l_i = \frac{1}{h_1}\det
				\begin{bmatrix}
					h_1 && h_{\nu-i+1} \\
					h_\nu && h_i
				\end{bmatrix}
				\mbox{ con } i=1,...,\nu-1
			\end{equation}
			Da qui ottengo una tabella triangolare. Se inoltre nel calcolo di $l_i$ ho un $h_1 = 0$ allora la tabella non è ben definita. Ci si ferma quando arriviamo a $n+1$ righe.
		\item Il sistema è AS. STABILE  sse
			\begin{itemize}
				\item La tabella è ben definita
				\item Gli elementi della prima colonna hanno lo stesso segno
			\end{itemize}
	\end{enumerate}
	\newpage
	\subsection{Sistemi non lineari tempo invarianti}
	\subsubsection{Linearizzazione di sistemi a tempo continuo}
	\begin{enumerate}
		\item Definire un movimento di equilibrio intorno al quale linearizzare. Dunque prendiamo $u(t) = \bar{u}$ e cerchiamo i punti $\bar{x}$ per i quali $f(\bar{x}, \bar{u}) = 0$
		\item Calcolo le matrici
			\begin{align}
				\frac{\partial f}{\partial x}(\bar{x}, \bar{u}) = A(\bar{x}, \bar{u}) &&
				\frac{\partial f}{\partial u}(\bar{x}, \bar{u}) = B(\bar{x}, \bar{u}) \\
				\frac{\partial g}{\partial x}(\bar{x}, \bar{u}) = C(\bar{x}, \bar{u}) &&
				\frac{\partial g}{\partial u}(\bar{x}, \bar{u}) = D(\bar{x}, \bar{u})
			\end{align}
		\item Descrivo le dinamiche del sistema, definendo
			\begin{align}
				\delta x = x - \bar{x} && \delta y = y - \bar{y} && \delta u = u - \bar{u}
			\end{align}
			\begin{equation}
				\begin{cases}
					\delta\dot{x} = A(\bar{x}, \bar{u})\delta x + B(\bar{x}, \bar{u})\delta u \\
					\delta\dot{y} = C(\bar{x}, \bar{u})\delta x + D(\bar{x}, \bar{u})\delta u
				\end{cases}
			\end{equation}
	\end{enumerate}	
	\paragraph{Stabilità degli equilibri}
	\begin{itemize}
		\item Se tutti gli autovalori di $A(\bar{x}, \bar{u})$ hanno $Re(\lambda_i) < 0$ allora il movimento di equilibrio $\bar{x}$ è ASINTOTICAMENTE STABILE (ricordando che si tratta di una stabilità locale).
		\item Se $\exists \lambda_i : Re(\lambda_i) > 0$ l'equilibrio $\bar{x}$ è instabile 
		\item Se $\forall \lambda_i : Re(\lambda_i) \leq 0$ ed $\exists \lambda_i : Re(\lambda_i) = 0$, non posso concludere nulla. In questo caso uso il metodo grafico, ovvero considero la funzione
		\begin{equation}
			\dot{x} = f(x, \bar{u}) = 
			\begin{cases}
				\mbox{se } f(x, \bar{u}) > 0 \rightarrow \dot{x} > 0 \\
				\mbox{se } f(x, \bar{u}) < 0 \rightarrow \dot{x} < 0 \\
			\end{cases}
		\end{equation}
	\end{itemize}
	\subsubsection{Linearizzazione di sistemi a tempo discreto}
	Il procedimento è il medesimo e si ottiene un sistema linearizzato del tipo
	\begin{equation}
	\begin{cases}
	\delta\dot{x}(t+1) = A(\bar{x}, \bar{u})\delta x(t) + B(\bar{x}, \bar{u})\delta u(t) \\
	\delta\dot{y}(t+1) = C(\bar{x}, \bar{u})\delta x(t) + D(\bar{x}, \bar{u})\delta u(t)
	\end{cases}
	\end{equation}
	\paragraph{Stabilità degli equilibri}
	Utilizziamo il criterio degli autovalori.\\
	Detti $\lambda_i$ per $i=1,...,\mu$ autovalori di $A$
	\begin{itemize}
		\item Se $\left|\lambda_i\right| < 1$, allora l'equilibrio $(\bar{x}, \bar{u})$ è AS. STABILE
		\item Se $\exists \lambda_i : \left|\lambda_i\right|>1$, l'equilibrio è INSTABILE
		\item Negli altri casi non si può dire nulla
	\end{itemize}

	\newpage
	\subsection{Trasformata di Laplace e funzioni di trasferimento}
	\subsubsection{Trasformate notevoli}
	\begin{center}
		\begin{tabular}{l | r}
			$\frac{1}{s}$ & $sca(t)$ \\
			$\frac{1}{s^2}$ & $ram(t)$ \\
			$\frac{1}{s^3}$ & $par(t)$ \\
			$\frac{s}{s^2 + \omega^2}$ & $cos(\omega t)$ \\
			$\frac{\omega}{s^2 + \omega^2}$ & $sin(\omega t)$ \\
			$\frac{1}{s + \alpha}$ & $e^{-\alpha t}sca(t)$ \\
		\end{tabular}	
	\end{center}
	\subsubsection{Proprietà dell'antitrasformata}
	\paragraph{Teorema della risposta finale}
	Data $F(s)$
	\begin{itemize}
		\item grado $D(s) >$ grado $N(s)$
		\item i poli di $F(s)$ sono o nulli o hanno $Re(s) < 0$
	\end{itemize}
	sotto queste ipotesi vale:
	\begin{equation}
		\lim_{t\to\infty} f(t) = \lim_{s\to 0} sF(s)
	\end{equation}
	\paragraph{Teorema della risposta iniziale}
	Data $F(s)$, se grado $D(s) >$ grado $N(s)$ allora vale:
	\begin{equation}
	\lim_{t\to 0} f(t) = \lim_{s\to\infty} sF(s)
	\end{equation}
	\subsubsection{Applicazione di Laplace e funzione di trasferimento}
	Dato un generico sistema del tipo
	\begin{equation}
		\begin{cases}
			\dot{x} = Ax + Bu\\
			\dot{y} = Cx + Du
		\end{cases}
	\end{equation}
	applicando la trasformata di Laplace otteniamo
	\begin{align}		
		X(s) = (sI-A)^{-1}X_0 + (sI-A)^{-1}BU(s) \\
		Y(s) = C(sI-A)^{-1}X_0 + (C(sI-A)^{-1}B+D)U(s) \\
		G(s) = C(sI-A)^{-1}B+D		
	\end{align}
	dove $G(s)$ è la funzione di trasferimento ed esprime il legame tra il segnale d'ingresso e quello d'uscita.
	\subsubsection{Schemi a blocchi}
	Date $G_1(s)$ e $G_2(s)$ funzioni di trasferimento di due sistemi $S_1$ e $S_2$, in base a come questi sono collegati, otteniamo un $G_{eq}(s)$ che è pari a
	\begin{itemize}
		\item Collegamento in serie: $G_{eq}(s) = G_1(s)G_2(s)$
		\item Collegamento in parallelo: $G_{eq}(s) = G_1(s) + G_2(s)$
		\item Retroazione: $G_{eq}(s) = \frac{G_1(s)}{1 + G_1(s)G_2(s)} = \frac{\mbox{f. di andata}}{1 + L(s)}$
	\end{itemize}

	\subsection{Tracciamento dei diagrammi di Bode}
	\begin{equation}
		G(s) = \frac{\mu}{s^g}
		\frac{\prod_i (1+s\tau_i)}{\prod_i (1+sT_i)}
		\frac{\prod_i (1+2\xi_i \frac{s}{\alpha_{ni}} + \frac{s^2}{\alpha^2_{ni}})}{\prod_i (1+2\xi_i \frac{s}{\omega_{ni}} + \frac{s^2}{\omega^2_{ni}})}
	\end{equation}
	\begin{equation}
	G(j\omega) = \frac{\mu}{(j\omega)^g}
	\frac{\prod_i (1+j\omega\tau_i)}{\prod_i (1+j\omega T_i)}
	\frac{\prod_i (1+2\xi_i \frac{j\omega}{\alpha_{ni}} - \frac{\omega^2}{\alpha^2_{ni}})}{\prod_i (1+2\xi_i \frac{j\omega}{\omega_{ni}} - \frac{\omega^2}{\omega^2_{ni}})}
	\end{equation}
	\subsubsection{Diagramma del modulo}
	\begin{flalign*}
		\left|G(j\omega)\right|_{dB} = 
		20\log_{10} \left|\mu\right| -
		20\log_{10} \left|\omega\right| + \\
		\sum_i 20\log_{10} \left|1 + j\omega\tau_i\right| -
		\sum_i 20\log_{10} \left|1 + j\omega T_i\right| + \\
		\sum_i 20\log_{10} \left| 1+2\xi_i \frac{j\omega}{\alpha_{ni}} - \frac{\omega^2}{\alpha^2_{ni}} \right| -
		20\log_{10} \left| 1+2\xi_i \frac{j\omega}{\omega_{ni}} - \frac{\omega^2}{\omega^2_{ni}} \right|
	\end{flalign*}
	\paragraph{Regole di tracciamento}
	\begin{enumerate}
		\item Tratto iniziale (prima di qualsiasi polo o zero) è una retta avente pendenza $-20g \mbox{ } dB/dec$, che nel punto $\omega_0 = 1$ assume il valore $20log_{10} \left|\mu\right|$
		\item In corrispondenza di:
		\begin{itemize}
			\item $\omega = \frac{1}{\tau_i}$ (zeri), la pendenza $\uparrow \mbox{ } m * 20 \mbox{ } dB/dec$
			\item $\omega = \frac{1}{T_i}$ (poli), la pendenza $\downarrow \mbox{ } m * 20 \mbox{ } dB/dec$
			\item $\omega = \omega_n$ (zeri C/C), la pendenza $\uparrow \mbox{ } 2 * m * 20 \mbox{ } dB/dec$
			\item $\omega = \omega_n$ (poli C/C), la pendenza $\downarrow \mbox{ } 2 * m * 20 \mbox{ } dB/dec$
		\end{itemize}
	\end{enumerate}
	\subsubsection{Diagramma della fase}
	\begin{align*}
		\arcangle{G(j\omega)} = \arcangle\mu - g90^o\\
		+ \sum_i \arcangle(1+j\omega\tau_i) - \sum_i \arcangle(1+j\omega T_i)\\
		+ \sum_i \arcangle( 1+2\xi_i \frac{j\omega}{\alpha_{ni}} - \frac{\omega^2}{\alpha^2_{ni}}) - 
		\sum_i \arcangle(1+2\xi_i \frac{j\omega}{\omega_{ni}} - \frac{\omega^2}{\omega^2_{ni}})		
	\end{align*}
	\paragraph{Regole di tracciamento}
	\begin{enumerate}
		\item Il tratto iniziale, fino al primo zero e polo, ha valore $\arcangle\mu-g90^o$
		\item çLa fase, in corrispondenza di:
		\begin{itemize}
			\item Polo reale:
				\subitem $T_i > 0 \implies \downarrow 90^o$
				\subitem $T_i < 0 \implies \uparrow 90^o$
			\item Zero reale:
				\subitem $\tau_i > 0 \implies \uparrow 90^o$
				\subitem $\tau_i < 0 \implies \downarrow 90^o$
			\item Poli C/C:
				\subitem $\xi > 0 \implies \downarrow 180^o$
				\subitem $\xi < 0 \implies \uparrow 180^o$
			\item Zeri C/C:
				\subitem $\xi > 0 \implies \uparrow 180^o$
				\subitem $\xi < 0 \implies \downarrow 180^o$
		\end{itemize}
	\end{enumerate}

	\subsection{Stabilità dei sistemi retroazionati}
	\paragraph{Criterio di Nyquist}
	Si definisca $P$ come il numero di poli di $L(s)$ con $Re > 0$. Si tracci il diagramma di Nyquist di $L(s)$ e si definisca con $N$ il numero di giri che il diagramma compie intorno al punto $-1$:
	\begin{itemize}
		\item positivamente se compiuti in senso antiorario
		\item negativamente se compiuti in senso orario
	\end{itemize}
	Se il diagramma passa per il punto $-1$, $N$ non è ben definita.\\
	Il sistema è AS. STABILE sse $N=P$
	\paragraph{Criterio di Bode}
	Sotto le ipotesi
	\begin{enumerate}
		\item $L$ strettamente propria
		\item $P=0$
		\item Il diagramma di Bode attraversa una e una sola volta l'asse a $0dB$, da sopra a sotto perche per l'asintotica stabilità deve valere  $\lim_{\omega\to\infty} L(j\omega) = -\infty$
	\end{enumerate}
	il sistema ad anello chiuso è AS. STABILE sse
	\begin{equation}
		\begin{cases}
			\mu > 0\\
			\varphi_m > 0
		\end{cases}
	\end{equation}
	
	\newpage
	\subsection{Prestazioni dei sistemi di controllo}
	\subsubsection{Prestazioni statiche}
	Definiamo le funzioni
	\begin{align*}
		F(s) = \frac{L(s)}{1+L(s)} && S(s) = \frac{1}{1+L(s)} && Q(s) = \frac{R(s)}{1+L(s)}
	\end{align*}
	otteniamo una tabella per la relazione tra i vari segnali
	\begin{center}
		\begin{tabular}{c | c | c | c}
			\hline
			$y(t)$ & $F(s)$ & $S(s)$ & $-F(s)$ \\ \hline
			$e(t)$ & $S(s)$ & $-S(s)$ & $F(s)$ \\ \hline
			$u(t)$ & $Q(s)$ & $-Q(s)$ & $-Q(s)$ \\ \hline
			/	   & $y^o(t)$ & $d(t)$ & $n(t)$ \\ \hline
		\end{tabular}
	\end{center}
	\subsubsection{Prestazioni dinamiche}
	Se vogliamo dei requisiti su
	\begin{align*}
		T_{ass} = T_s && S\% \leq  s
	\end{align*}
	otteniamo delle specifiche su $\varphi_m$ e $\omega_c$
	\paragraph{Esempio pratico}
	Se voglio una $S\% \leq  s$ calcolo
	\begin{align}
		100e^{\frac{-\xi_F\pi}{\sqrt{1-\xi_F}}} \implies \xi_F \geq \sqrt{\frac{\rho}{1+\rho}} &&
		\mbox{ dove }
		\rho = \frac{\ln^2(\frac{s}{100})}{\pi^2}
	\end{align}
	%Sapendo che
	%\begin{equation}
	%	\xi_F \cong \frac{\varphi_m}{100} \implies \varphi_m \geq 100\xi_F = \varphi_{m,min}
	%\end{equation}
	Considero poi due casi, in modo da distinguere i casi di approssimazione di ordine 1 e quella di ordine 2
	\begin{itemize}
		\item Se $\varphi_m \leq 75^o$
			\begin{equation}
				T_{ass} \cong \frac{5}{\xi_F \omega_c} \leq T_s \implies \omega_c\cdot\varphi_m \geq \frac{5 \cdot 100}{s}
			\end{equation}
		\item Se $\varphi_m > 75^o$
			\begin{equation}
				T_{ass} \cong \frac{5}{\omega_c} \leq s \implies \omega_c \geq \frac{s}{5}
			\end{equation}
	\end{itemize}

	\newpage
	\subsection{Regolatore discretizzato}
	\paragraph{Passaggio da R(s) a R*(s)}
	Ci sono tre modi per passare da una funzione di un regolatore a tempo continuo a una a tempo discreto:
	\begin{itemize}
		\item Eulero Esplicito
			\begin{equation}
				s = \frac{z-1}{T_s}
			\end{equation}
		\item Eulero Implicito
			\begin{equation}
				s = \frac{z-1}{zT_s}
			\end{equation}
		\item Tustin
			\begin{equation}
				s = \frac{2}{T_s}\frac{z-1}{z+1}
			\end{equation}
	\end{itemize}
	\paragraph{Variazione del margine di fase}
	\begin{equation}
		\Delta\varphi_m = -\frac{T_s}{2}\cdot\omega_c\cdot\frac{180}{\pi}
	\end{equation}
	\paragraph{Calcolo del tempo di campionamento per il rispetto di un margine di fase reale}
	Considerando:
	\begin{itemize}
		\item Il ritardo corrispondente alla cascata campionatore+mantenitore pari a $\tau_S = T_S/2$, il corrispondente contributo al margine di fase (in gradi) è $-\tau_S\cdot\omega_c\cdot\frac{180}{\pi}$
		\item Il contributo del margine di fase (in gradi) dato dal ritardo di elaborazione è pari a $-\tau_{EL}\omega_c\frac{180}{\pi}$
		\item Il filtro passabasso antialiasing ha banda $[0, \omega_{AA}]$ con $\omega_{AA}>>\omega_c$. Lo sfasamento conferito dal filtro al margine di fase è pari a $-\arctan(\omega_c/\omega_{AA})$
	\end{itemize}
	Da qui otteniamo
	\begin{equation}
		\varphi_m^{REALE} = \varphi_m^o - (\frac{T_S}{2} + \tau_{EL})\cdot\omega_c\frac{180}{\pi} - \arctan(\omega_c/\omega_{AA})
	\end{equation}
	Sostituendo tutti i valori e quindi anche il margine di fase reale desiderato si ottiene il valore $T_S$.
\end{document}